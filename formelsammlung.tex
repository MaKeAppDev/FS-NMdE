% Formelsammlung Numerische Methoden der Elektrotechnik
%
% Geschrieben im SS 2014 an der TU München
% von Markus Hofbauer, Kevin Meyer und Benedikt Schmidt für LaTeX4EI
% based on template from www.latex4ei.de
% Kontakt: latex@kevin-meyer.de oder via Kontaktformular auf http://latex4ei.de
% Aktuelle Versionen auf https://makeappdev.github.io/TUM-Projekte/

% Dokumenteinstellungen
% ======================================================================
\documentclass[german]{latex4ei/latex4ei_sheet}

\usepackage[european]{circuitikz}
\usepackage{tabularx}
\usepackage{multirow}
\usetikzlibrary{arrows, calc, intersections}
\usepackage{calc}

% Für code
\definecolor{COMMENTGREEN}{HTML}{228B22}
\definecolor{MATLABBACKGROUND}{HTML}{FCFCDC}
\lstset{ %
language=Matlab,						% choose the language of the code
%basicstyle=\ttfamily,					% the size of the fonts that are used for the code
%emphstyle=\color{yellow}\ttfalily,
%keywordstyle=\color{blue}\ttfamily,
%stringstyle=\color{magenta}\ttfamily,
%commentstyle=\color{COMMENTGREEN}\ttfamily,
xleftmargin=10.3pt,						% distance to margin left
xrightmargin=-3pt,						% distance to margin right
%linewidth=\widthof{\begin{sectionbox}}}		% this would be better instead of left and right margin
%aboveskip=0.6\baselineskip,
%belowskip=0\baselineskip,
numbers=left,                   		% where to put the line-numbers
%framexleftmargin=1.5em,					% distance to xleftmargin
%numberstyle=\ttfamily\footnotesize,		% the size of the fonts that are used for the line-numbers
%stepnumber=1,							% the step between two line-numbers. If it is 1 each line will be numbered
%numbersep=5pt,							% how far the line-numbers are from the code
%backgroundcolor=\color{MATLABBACKGROUND},	% choose the background color. You must add \usepackage{color}
%showspaces=false,						% show spaces adding particular underscores
%showstringspaces=false,					% underline spaces within strings
%showtabs=false,							% show tabs within strings adding particular underscores
%frame=single,							% adds a frame around the code (Box) (Für top und bottom rule set option to "lines")
%rulecolor=\color{gray},					% color of framebox rule
%tabsize=4,								% sets default tabsize to 2 spaces
%captionpos=b,							% sets the caption-position to bottom
breaklines=true,						% sets automatic line breaking
breakatwhitespace=false,				% sets if automatic breaks should only happen at whitespace
escapeinside={\%*}{*)}					% if you want to add a comment within your code
}

% tabularx definition
\newcolumntype{C}{>{\centering\arraybackslash}X}
\newcolumntype{L}{@{\extracolsep\fill}X}

% SI-Zahlen mit Komma als Dezimaltrenner
\sisetup{locale=DE}

% SI-Einheiten
\DeclareSIUnit\voltampere{VA}
\DeclareSIUnit\var{Var}
\DeclareSIUnit\newtonmeter{Nm}
\DeclareSIUnit\voltsecond{Vs}
\DeclareSIUnit\amperesecond{As}

\DeclareMathOperator*{\argmin}{arg\,min}
\DeclareMathOperator{\cond}{cond}
\DeclareMathOperator{\rang}{rang}

\DeclareMathOperator{\Bild}{Bild}
\DeclareMathOperator{\defect}{def}

% Title
\title{Numerische Methoden der Elektrotechnik}
\author{Markus Hofbauer, Kevin Meyer und Benedikt Schmidt}
\myemail{latex@kevin-meyer.de}
\mywebsite{https://makeappdev.github.io/TUM-Projekte/}

% Dokumentbeginn
% ======================================================================
\begin{document}

\IfFileExists{git.id}{\input{git.id}}{}
\ifdefined\GitRevision\mydate{\GitNiceDate\ (git \GitRevision)}\fi

\maketitle

\section{Grundlagen}
\begin{sectionbox}
	\subsection[Numerik]{Numerik $f(x) = y \quad \Ra \quad \tilde f(\tilde x) = \tilde y$}
	liefert eine zahlenmäßige Lösung eines Problems mit einem Algorithmus\\[-1em]
	\begin{symbolbox}
		\begin{tabular*}{\columnwidth}{@{\ }l@{\ }l@{\qquad}l@{\ }l@{}}
			$f$ & Mathematisches Problem & $\tilde f$ & Numerischer Algorithmus\\
			$x$ & exakte Problemdaten & $\tilde x$ & gerundete Problemdaten\\
			$y$ & exaktes Ergebnis & $\tilde y$ & gerundetes Ergebnis\\
		\end{tabular*}
	\end{symbolbox}

	\subsection{Fehlertypen}
	\textbf{Datenfehler:} Eingabedaten aus ungenauer Messung\\
	\textbf{Verfahrensfehler:} Diskretisierung von Gleichungen, Endliche Iteration\\
	\textbf{Rundungsfehler:} (Zwischen-)Ergebnisse nur mit Maschinengenauigkeit

	\subsection{Numerische Qualitätsmerkmale}
	\textbf{Konsistenz:} Wie gut löst das Verfahren tatsäch. das Problem $\tilde f(x)\!\ra\!y$?\\
	Residuum $R < C \cdot h^p$ \quad Schrittweite $h$, Konsistenzordn. $p$\\
	\textbf{Kondition:} Wie stark schwankt das Problem bei Störung $f(\tilde x)\!\ra\!y$ ?\\[0.1em]
	\textbf{Stabilität:} Wie stark schwankt das Verfahren bei Störung $\tilde f(\tilde x)\!\ra\!\tilde f(x)$
	\textbf{Konvergenz:} Algorithmus stabil und konsistent:  $\tilde f(\tilde x)\!\ra\!\tilde f(x)\!\ra\!y$
\end{sectionbox}

\begin{sectionbox}
\subsection{Spektralradius}
	\textbf{Spektralradius $\rho(\ma A)$ einer Matrix $\ma A$:} Betragsmäßig größter Eigenwert. \\
	Konvergenzbeweis aller Verfahren: Gershgorinkreise um die Null mit $r \le 1$
	\begin{equation*}
		\rho(\ma{A}) = \max_i \abs{\lambda_i}
	\end{equation*}
\subsection{Diagonaldominanz}
Diagonalelemente sind größer als die restlichen Elemente der selben Zeile:\\
$\ma{A} = (a_{ij}) \begin{aligned} \text{diagonaldominant} \\ \text{strikt diagonaldominant}\end{aligned} \Leftrightarrow \abs{a_{ii}} \begin{aligned} \ge \\ > \end{aligned} \sum\limits_{j = 1, j \ne i}^n \abs{a_{ij}}\ \forall i$
\subsection{Definitheit}
$\ma{A} \begin{aligned} \text{positiv definit} \\ \text{ positiv semidefinit}\end{aligned} \Leftrightarrow \lambda_i \begin{aligned} > \\ \ge \end{aligned} 0\ \forall i \qquad\text{bzw.}\qquad\vec{x}^T \ma{A} \vec{x} \begin{aligned} > \\ \ge \end{aligned} 0\ \forall \vec{x} \ne 0$
\end{sectionbox}

\begin{sectionbox}
	\subsection{Kondition}
	Ein Maß wie stark sich Eingabefehler auf die Ausgabe auswirken.
	$\kappa_{\ir abs}(x) = \abs{f'(x)}$ \qquad\qquad $\kappa_{\ir rel}(x) = \abs{\frac{f'(x)}{\frac{f(x)}{x}}} = \frac{\abs{f'(x)} \cdot \abs{x}}{\abs{f(x)}}$\\
	Falls $\kappa_{\ir rel} \ll 100$: gute Konditionierung\\
	Verkettung $h = g(f(x))$ \quad $\kappa^h_{\ir abs}(x) = \kappa^g_{\ir abs}(f(x))\kappa^f_{\ir abs}(x)$\\
	\begin{equation*}
		\text{für } \vec y = \ma A \vec x \qquad \Ra \quad \cond(\ma{A}) = \norm{\ma{A}^{-1}} \cdot \norm{\ma{A}}
	\end{equation*}
	$\cond(\ma{A}) \ra \infty$ schlecht, $\cond(\ma{A}) \ra 1$ gut

	\subsection{Fehler}
	Absolut: $\norm{\tilde f(x) - f(x)}$ \qquad\qquad Relativ: $\frac{\norm{\tilde f(x) - f(x)}}{\norm{f(x)}}$

	\subsection[Residuum]{Residuum $\vec r = \vec b - \ma A\vec x$}
	bezeichnet die Abweichung vom gewünschten Ergebnis, wenn Näherungslösungen eingesetzt werden. $\vec r$ klein $\Rightarrow$ rel. Fehler $\ll 1$.
\end{sectionbox}

\begin{sectionbox}
\subsection{Parametrisierung einer Geraden}
\begin{tabularx}{\columnwidth}{CCC}
\multirow{2}{*}{$g(x) = a x + b$} & $y_1 = g(x_1)$ & $a = \frac{y_1 - y_2}{x_1 - x_2}$\\
& $y_2 = g(x_2)$ & $b = \frac{x_1 y_2 - x_2 y_1}{x_1 - x_2}$
\end{tabularx}

\subsection{Schnittpunkt zweier Geraden}
\begin{tabularx}{\columnwidth}{CC}
$a_{11}x_1 + a_{12}x_2 = b_1$ & $x_1 = \frac{a_{22}b_1 - a_{12}b_2}{a_{11}a_{22}-a_{12}a_{21}}$\\
$a_{21}x_1 + a_{22}x_2 = b_2$ & $x_2 = \frac{-a_{21}b_1 + a_{11}b_2}{a_{11}a_{22}-a_{12}a_{21}}$
\end{tabularx}
\end{sectionbox}

\begin{sectionbox}
	\subsection[Matrizen]{Matrizen $\ma A \in\mathbb{K}^{m \times n}$}
	\begin{tabularx}{\columnwidth}{LX}
	$(\ma A + \ma B)^\top = \ma A^\top + \ma B^\top$ & $(\ma A \cdot \ma B)^\top = \ma B^\top \cdot \ma A^\top$\\
	${(\ma A^\top)}^{-1} = {(\ma A^{-1})}^\top$ & $(\ma A \cdot \ma B)^{-1} = \ma B^{-1}\ma A^{-1}$
	\end{tabularx}

	\subsubsection{Dimensionen}
	\begin{tablebox}{ll}
	Bildraum & Nullraum \\ \mrule
	$\Bild \ma A = \iset{\ma A \vec x}{\vec x \in \K^n }$ & $\ker\ma A = \iset{\vec x \in \K^n}{\ma A \vec x = \vec 0}$\\
	$\rang \ma A = \dim(\Bild \ma A)$ & $\defect \ma A = \dim(\ker \ma A)$\\
	\end{tablebox}
	$\rang \ma A = r$ ist Anzahl. lin. unab. Spaltenvektoren.\\
	$\ma A$ erzeugt $\mathbb K \Leftrightarrow r = n$ \qquad $\ma A$ ist Basis von $\mathbb K \Leftrightarrow r = n = m$\\
	$\dim \mathbb K = n = \rang\ma A + \dim\ker\ma A$ \qquad $\rang\ma A = \rang\ma A^\top$

	\subsubsection{Quadratische Matrizen $A \in \mathbb{K}^{n \times n}$}
	regulär/invertierbar/nicht-singulär $\Leftrightarrow \det (\ma A) \ne 0 \Leftrightarrow \rang\ma A = n$\\
	singulär/nicht-invertierbar $\Leftrightarrow \det (\ma A) = 0 \Leftrightarrow \rang\ma A \ne n$\\
	\begin{tabular*}{\columnwidth}{@{\extracolsep\fill}ll}
		\multicolumn{2}{@{\extracolsep\fill}l}{orthogonal $\Leftrightarrow \ma A^\top=\ma A^{-1} \Ra \det(\ma A) = \pm 1$}\\
		symmetrisch: $\ma A=\ma A^\top$ & schiefsymmetrisch: $\ma A=-\ma A^\top$\\
		hermitsch: $\ma A=\overline{\ma A}^\top$ & unitär:$\ma A^{-1} = \overline{\ma A}^\top$
	\end{tabular*}

	\subsubsection[Determinante]{Determinante von $\ma A\in \mathbb K^{n\times n}$: $\det(\ma A)=|\ma A|$}
	$\det\mat{ \ma A & \ma 0 \\ \ma C& \ma D }= \det\mat{ \ma A & \ma B \\ \ma 0 & \ma D } = \det(\ma A)\det(\ma D)$ \\
	\begin{tabular*}{\columnwidth}{@{\extracolsep\fill}ll}
	$\det(\ma A) = \det(\ma A^T)$ & $\det(\ma A^{-1}) = \det(\ma A)^{-1}$
	\end{tabular*}
	$\det(\ma A\ma B) = \det(\ma A)\det(\ma B) = \det(\ma B)\det(\ma A) = \det(\ma B\ma A)$\\
	Hat $\ma A$ 2 linear abhäng. Zeilen/Spalten $\Rightarrow |\ma A|=0$ \\
	Entwicklung. n. $j$ter Zeile: $|\ma A|=\sum\limits_{i=1}^n (-1)^{i+j} \cdot a_{ij} \cdot |\ma A_{ij}|$\\

	\subsubsection{Eigenwerte $\lambda$ und Eigenvektoren $\underline v$}
	\framebox[\columnwidth]{\large $\ma A \vec v = \lambda \vec v$ \qquad $\det \ma A = \prod \lambda_i$ \qquad $\Sp \ma A = \sum a_{ii} = \sum \lambda_i$}\\
	Eigenwerte: $\det(\ma A - \lambda \ma 1) = 0$ Eigenvektoren: $\ker(\ma A - \lambda_i \ma 1) = \vec v_i$\\
	EW von Dreieck/Diagonal Matrizen sind die Elem. der Hauptdiagonale.

	\subsubsection{Spezialfall $2 \times 2$ Matrix $A$}
	\parbox{3cm}{ $\det(\ma A) = ad-bc$ \\ $\Sp(\ma A) = a+d$ } $\mat{a & b\\ c & d}^{-1} = \frac{1}{\det \ma A} \mat{d & -b\\ -c& a}$\\
	$\lambda_{1/2} = \frac{\Sp \ma A}{2} \pm \sqrt{ \left( \frac{\mathrm{sp} \ma A}{2} \right)^2 - \det \ma A }$

	\subsubsection{Spezielle Matrizen}
	Diagonalmatrix $\ma D$:  $\det \ma D = \prod d_{i}$\\
	$\ma D^{-1} =\operatorname{diag} \left(d_1, \dots, d_n\right)^{-1} = \operatorname{diag} \left(d_1^{-1}, \dots, d_n^{-1}\right)$\\
\end{sectionbox}

\begin{sectionbox}
	\subsection{Norm $|| \cdot ||$}
	Definition: Zahl, die die „Größe“ eines Objekts $\mathcal X$ beschreibt.\\
	Jede Norm muss folgende 3 Axiome erfüllen::
	\begin{enumerate}
		\item Definitheit: $\norm{\mathcal X} \ge 0$ mit $\norm{\mathcal X} = 0 \Leftrightarrow \mathcal X = 0$
		\item absolute Homogenität:	$\norm{\alpha\cdot \mathcal X} = |\alpha| \cdot \norm{\mathcal X}$ \qquad ($\alpha$ ist skalar)
		\item Dreiecksungleichung: $\norm{\mathcal X + \mathcal Y} \leq \norm{\mathcal X} + \norm{\mathcal Y}$
	\end{enumerate}

	\subsubsection[Vektornormen]{Vektornormen: ($\vec x \in \K^n, \sum$ von $i=0$ bis $n$)}
	\begin{tablebox}{l@{\ }ll@{\ }l}
		Summen & $\norm{\vec x}_1 = \sum |x_i|$ & Euklidische & $\norm{\vec x}_2 = \sqrt{\sum |x_i|^2}$\\
		Maximum & $\norm{\vec x}_\infty = \max |x_i|$ & Alg. p-Norm & $\norm{\vec x}_p = \left( \sum |x_i|^p \right)^{1\!/\!p}$\\
	\end{tablebox}
\end{sectionbox}

\begin{sectionbox}
	\subsubsection[Matrixnormen]{Matrixnormen ($\ma A \in \K^{m \times n}, i\in[0,m], j\in[0,n]$)}
	Für Matrixnormen gilt zu den 3 Standard Axiomen zusätzlich:
	\begin{enumerate} \setcounter{enumi}{3}
		\item Submultiplikativität: $\norm{\ma A + \ma B} \leq \norm{\ma A} \cdot \norm{\ma B}$
	\end{enumerate}

	\begin{tablebox}{lr@{ = }l}
	Gesamtnorm $\left(\norm{\ma{A}}_M = \frac{\norm{\ma{A}}_G}{\sqrt{mn}}\right)$ & $\norm{\ma{A}}_G$ & $\sqrt{mn}\cdot\underset{i,j}{\max}\abs{a_{ij}}$\\
	Zeilennorm (max Zeilensumme) & $\norm{\ma{A}}_\infty$ & $\underset{i}{\max}\sum\limits_{j=1}^n\abs{a_{ij}}$ \\
	Spaltennorm (max Spaltensumme) & $\norm{\ma{A}}_1$ & $\underset{j}{\max}\sum\limits_{i=1}^n\abs{a_{ij}}$ \\
	$\underset{\text{euklidische Norm}}{\text{Frobeniusnorm}}$ $(||\ma{I}||_E = \sqrt{n})$ & $\norm{\ma{A}}_E$ & $\sqrt{\sum\limits_{i=1}\sum\limits_{j=1}\abs{a_{ij}}^2}$\\
	Spektralnorm, Hilbertnorm & $\norm{\ma{A}}_\lambda$ & $\sqrt{\lambda_\text{max}(\ma{A}^\top\cdot\ma{A})}$\\
	\end{tablebox}
\end{sectionbox}

\begin{sectionbox}
\subsection{Jacobi-Matrix}
$\vec f: \mathbb{R}^n \rightarrow \mathbb{R}^m$
\begin{equation*}
	\frac{\partial }{\partial \vec x} \vec f(\vec x) = \ma J_f (x) =
	\begin{bmatrix}
		\frac{\partial f_1}{\partial x_1} & \hdots & \frac{\partial f_1}{\partial x_n} \\
		\vdots & & \vdots \\
		\frac{\partial f_m}{\partial x_1} & \hdots & \frac{\partial f_m}{\partial x_n}
	\end{bmatrix} =
	\begin{bmatrix}
		(\nabla f_1)^T \\
		\vdots \\
		(\nabla f_m)^T
	\end{bmatrix}
\end{equation*}
\end{sectionbox}

\begin{sectionbox}
\subsection{Wichtige Formeln}
\setlength{\tabcolsep}{0pt}
\begin{tablebox}{@{\extracolsep\fill}ll@{}}
Dreiecksungleichung: &$\big|\! \abs{x}- \abs{y}\!\big| \le \abs{x \pm y} \le \abs{x} + \abs{y}$\\
Cauchy-Schwarz-Ungleichung: & $\left| \vec x^\top \bdot \vec y \right| \le \| \vec x\| \cdot \| \vec y\|$ \\
Bernoulli-Ungleichung: & $(1+x)^n \ge 1+nx$\\ \cmrule
Arithmetische Summenformel &  $\sum \limits_{k=1}^{n} k = \frac{n (n+1)}{2} $ \\
Geometrische Summenformel &  $ \sum \limits_{k=0}^{n} q^k = \frac{1 - q^{n+1}}{1-q}$ \\
Binomialkoeffizient & $\binom nk = \binom n{n-k} = \frac{n!}{k! \cdot (n-k)!}$\\
\end{tablebox}
\end{sectionbox}

\begin{sectionbox}
\subsection{Sinus, Cosinus \quad $\sin^2(x) \bs + \cos^2(x) = 1$}
\qquad $e^{\j x} = \cos (x) + \j \cdot \sin(x)$
\setlength{\tabcolsep}{4pt}
\begin{tablebox}{@{\extracolsep\fill}c|c|c|c|c||c|c|c|c@{}}
$x$ & $0$ & $\pi / 6$ & $\pi / 4$ & $\pi / 3$ & $\frac{1}{2}\pi$ & $\pi$ & $\frac{3}{2}\pi$ & $2 \pi$ \\
$\varphi$ & $\SI{0}{\degree}$ & $\SI{30}{\degree}$ & $\SI{45}{\degree}$ & $\SI{60}{\degree}$ & $\SI{90}{\degree}$ & $\SI{180}{\degree}$ & $\SI{270}{\degree}$ & $\SI{360}{\degree}$ \\ \cmrule
$\sin$ & $0$ & $\frac{1}{2}$ & $\frac{1}{\sqrt{2}}$ & $\frac{\sqrt 3}{2}$ & $1$ & $0$ & $-1$ & $0$ \\
$\cos$ & $1$ & $\frac{\sqrt 3}{2}$ & $\frac{1}{\sqrt 2}$ & $\frac{1}{2}$ & $0$ & $-1$ & $0$ & $1$ \\
$\tan$ & $0$ & $\frac{\sqrt{3}}{3}$ &	$1$	&	$\sqrt{3}$ & $\pm \infty$ & $0$ & $\mp \infty$ & $0$\\
\end{tablebox}
\begin{tabular*}{\columnwidth}{@{\extracolsep\fill}ll@{}}
	Additionstheoreme &  Stammfunktionen\\
 	$\cos (x - \frac{\pi}{2}) = \sin x$ & $\int x \cos(x) \diff x = \cos(x) + x \sin(x)$\\
 	$\sin (x + \frac{\pi}{2}) = \cos x$ & $\int x \sin(x) \diff x = \sin(x) - x \cos(x)$\\
 	$\sin 2x = 2 \sin x \cos x $  & $\int \sin^2(x) \diff x = \frac12 \bigl(x - \sin(x)\cos(x) \bigr)$\\
 	$\cos 2x = 2\cos^2 x - 1$  & $\int \cos^2(x) \diff x = \frac12 \bigl(x + \sin(x)\cos(x) \bigr)$\\
 	$\sin(x) = \tan(x)\cos(x)$ & $\int \cos(x)\sin(x) = -\frac12 \cos^2(x)$ \\
\end{tabular*}\\[1em]
	\textbf{Sinus/Cosinus Hyperbolicus} $\sinh, \cosh$\\
	\begin{tabular*}{\columnwidth}{@{\extracolsep\fill}ll@{}}
	$\sinh x = \frac{1}{2}(e^x -e^{-x})= - \j \, \sin(\j x)$ & $\cosh^2 x  \bs - \sinh^2 x = 1$\\
	$\cosh x  = \frac{1}{2}(e^x +e^{-x})= \cos(\j x)$ & $\cosh x + \sinh x = e^{x}$\\
	\end{tabular*}\\
	\textbf{Kardinalsinus} $\mathrm{si}(x) = \frac{\sin(x)}{x}$ \qquad \textbf{genormt:} $\sinc(x) = \frac{\sin(\pi x)}{\pi x}$
\end{sectionbox}

\begin{sectionbox}
	\subsection{Integrale $\int e^x\;\mathrm dx = e^x = (e^x)'$}
	\begin{tabularx}{\columnwidth}{lX}
	Partielle Integration: & $\int uw'=uw-\int u'w$\\
	Substitution: & $\int f(g(x)) g'(x)\diff x=\int f(t)\diff t$
	\end{tabularx}
	\begin{tablebox}{@{\hspace{5mm}}c@{\extracolsep\fill}c@{\extracolsep\fill}c@{\hspace{5mm}}}
	\renewcommand{\arraystretch}{1.6}
		$F(x)$ & $f(x)$ & $f'(x)$ \\ \cmrule
		$\frac{1}{q+1}x^{q+1}$ & $x^q$ & $qx^{q-1}$ \\
		\raisebox{-0.2em}{$\frac{2\sqrt{ax^3}}{3}$} & $\sqrt{ax}$ & \raisebox{0.2em}{$\frac{a}{2\sqrt{ax}}$}\\
		$x\ln(ax) -x$ & $\ln(ax)$ & $\textstyle \frac{1}{x}$\\
		$\frac{1}{a^2} e^{ax}(ax- 1)$ & $x \cdot e^{ax}$ & $e^{ax}(ax+1)$ \\
		$\frac{a^x}{\ln(a)}$ & $a^x$ & $a^x \ln(a)$ \\
		$-\cos(x)$ & $\sin(x)$ & $\cos(x)$\\
		$\cosh(x)$ & $\sinh(x)$ & $\cosh(x)$\\
		$-\ln |\cos(x)|$ & $\tan(x)$ & $\frac{1}{\cos^2(x)}$ \\
	\end{tablebox}

	\begin{tabularx}{\columnwidth}{Ll@{}}
	\multicolumn{2}{c}{$\int e^{at} \sin(bt) \diff t = e^{at} \frac{a \sin(bt) + b \cos(bt)}{a^2 + b^2}$}\\
	$\int \frac{\diff t}{\sqrt{at+b}} = \frac{2 \sqrt{at+b}}{a}$ & $\int t^2 e^{at} \diff t = \frac{(ax-1)^2+1}{a^3} e^{at}$\\
	$\int t e^{at} \diff t = \frac{at-1}{a^2} e^{at}$ & $\int x e^{ax^2} \diff x = \frac{1}{2a} e^{ax^2}$\\
	\end{tabularx}
\end{sectionbox}

\begin{sectionbox}
\subsection{Exponentialfunktion und Logarithmus}
\begin{tabular*}{\columnwidth}{l@{\extracolsep\fill}ll}
	$a^x = e^{x \ln a}$ & $\log_a x = \frac{\ln x}{\ln a}$ & $\ln x \le x -1$\\
	$\ln(x^{a}) = a \ln(x)$ & $\ln(\frac{x}{a}) = \ln x - \ln a$ & $\log(1) = 0$\\
\end{tabular*}

\subsection{Reihen}
\begin{tabularx}{\columnwidth}{CCC}
$\underset{\text{Harmonische Reihe}}{\sum\limits_{n=1}^\infty \frac{1}{n} \ra \infty}$ & $\underset{\text{Geometrische Reihe}}{\sum\limits_{n=0}^\infty q^n \stackrel{|q|<1}= \frac{1}{1-q}}$ & $\underset{\text{Exponentialreihe}}{\sum\limits_{n = 0}^{\infty} \frac{z^n}{n!} = e^z}$
\end{tabularx}
\end{sectionbox}

\section{Lösung nichtlinearer Gleichungen}
Exakte Lösung $x*$, Fehler $\epsilon=x-x*$

\begin{sectionbox}
	\subsection{Iterationsverfahren (Nullstellensuche)}
	Problem: $f(x) = 0, f(x)$ stetig in $[a,b]$ und $f(a) \cdot f(b) < 0$\\
	Gesucht: $x^*:f(x^*)=0, a \le x^* \le b$

	Konvergenz: $ε^{(k+1)} = \frac{1}{2} ε^{(k)} = \left( \frac12 \right)^{k+1} ε^{(0)} $\\
	Iterationsschritte bis $ε < τ$: $k = \ceil{\ld\left(\frac{ε^{(0)}}{τ}\right)}$
\end{sectionbox}

\begin{sectionbox}
\subsection{Fixpunktiteration (alg. Iterationsverfahren)}
Jedes Problem $f(x) = g(x)$ lässt sich als Fixpunktproblem schreiben:\\
$x^* = Φ(x^*) := f(x) - g(x) + x$
\begin{equation*}
	x^{(k + 1)} = \Phi(x^{(k)})
\end{equation*}
\begin{equation*}
	x^\star = \Phi(x^\star)
\end{equation*}
Falls $\abs{\Phi'(x^\star)} < 1 \Rightarrow$ Konvergenz bzw. stabiler Fixpunkt \\
Falls $0 < \abs{\Phi'(x^\star)} < 1 \Rightarrow$ lineare Konvergenz mit
\begin{equation*}
	\varepsilon^{(k + 1)} \approx \Phi'(x^\star) \varepsilon^{(k)}
\end{equation*}
Falls $\Phi'(x^\star) = 0$ und $\Phi''(x^\star) \ne 0 \Rightarrow$ quadratische Konvergenz \\
\textbf{Allgemein:} Konvergenzordnung $n \Leftrightarrow$\\ $\Phi'(x^\star) = \Phi''(x^\star) = \ldots = \Phi^{(n - 1)}(x^\star) = 0$ und $\Phi^{(n)}(x^\star) \ne 0$
\end{sectionbox}

\begin{sectionbox}
\subsection{Newton-Raphson}
	Funktion durch Gerade annähern und Nullstelle bestimmen. An dieser Stelle den Vorgang wiederholen. Nur lokale Konvergenz\\
\textbf{Ausgangsproblem:}
\begin{equation*}
	f(x) = 0
\end{equation*}
\begin{equation*}
	x^{(k + 1)} = x^{(k)} - \frac{f(x^{(k)})}{f'(x^{(k)})}  =: Φ\left(x^{(k)}\right)
\end{equation*}
\end{sectionbox}

\begin{sectionbox}
\subsubsection{Konvergenz}
Falls $\abs{\Phi'(x^\star)} < 1 \Rightarrow$ Konvergenz bzw.\ stabiler Fixpunkt

Falls $f'(x^\star) \ne 0$ (einfache Nullstelle) $\Rightarrow$ quadratische Konvergenz mit
\begin{equation*}
	\varepsilon^{(k + 1)} = \frac{1}{2} \frac{f''(x^\star)}{f'(x^\star)} {\varepsilon^{(k)}}^2
\end{equation*}
Falls $f'(x^\star) = 0$ (Nullstellengrad $n > 1$) $\Rightarrow$ lineare Konvergenz mit
\begin{equation*}
\textbf{Konvergenzfaktor} = \frac{n-1}{n}
\end{equation*}

\subsubsection{Sekanten-Methode}
Falls die Auswertung von $f'(x)$ vermieden werden soll:
\begin{equation*}
	x^{(k + 1)} = x^{(k)} - \frac{f(x^{(k)}) \left( x^{(k)} - x^{(k - 1)} \right)}{f(x^{(k)}) - f(x^{(k - 1)})}
\end{equation*}

\subsubsection{Mehrdimensional}
\textbf{Theoretisch:}
\begin{equation*}
	\vec x^{(k + 1)} = \vec x^{(k)} - \ma J^{-1}_{\vec f}\left (\vec x^{(k)}\right ) \vec f(x^{(k)})
\end{equation*}
\textbf{Praktisch:}
\begin{equation*}
	\ma J_{\vec f}\left (\vec x^{(k)}\right ) \vec x^{(k + 1)} = \ma J_{\vec f}\left (\vec x^{(k)}\right )\cdot \left (\vec x^{(k)} - \vec f(x^{(k)}\right )
\end{equation*}
\end{sectionbox}

\section{Lösung linearer Gleichungssysteme}
\begin{sectionbox}
\textbf{Ausgangsproblem:}
\begin{equation*}
	\ma{A} \vec{x} = \vec{b}
\end{equation*}
\begin{tablebox}{ll}
	$\ma{A} = \ma{M} - \ma{N}$ & Systemmatrix\\
	$\ma{D}$ & Diagonalmatrix \texttt{diag(diag(}$\ma{A}$\texttt{))}\\
	$\ma{L}$ & Linke untere Dreiecksmatrix \texttt{tril(}$\ma{A},-1$\texttt{)}\\
	$\ma{U}$ & Rechte obere Dreiecksmatrix \texttt{triu(}$\ma{A},1$\texttt{)}\\
\end{tablebox}
$\ma A = \ma D + \ma L + \ma U$, keine LR-Zerlegung!
\end{sectionbox}

\begin{sectionbox}
	\subsection{Allgemeines Iterationsverfahren}
	Mit beliebiger, invertierbarer Matrix $\ma C \in \R^{n \times n}$ gilt Umformung:\\
	$\ma A \vec x\!=\!\vec b \Leftrightarrow \ma C \vec x = \ma C \vec x - \ma A \vec x + \vec b \Leftrightarrow \vec x = (\ma 1 - \ma C^{-1} \ma A) \vec x + \ma C^{-1} \vec b$\\
	Wähle $\ma K = (\ma 1 - \ma C^{-1} \ma A)$ (alles vor dem $\vec x$)\\
	Verfahren konvergiert allgemein, wenn Spektralradius $\rho(\ma{K}) < 1$\\
\end{sectionbox}

\begin{sectionbox}
\subsection{Jacobi-Verfahren}
Konvergiert falls $\rho(\ma{K}_j) < 1$ oder falls $\ma{A}$ strikt diagonaldominant\\
Spektralradius $\rho(\ma{K}_j) = \max |\lambda_i(\ma A)|$ mit $\lambda_i$ EW.
\begin{equation*}
	\ma{K}_j = \ma{D}^{-1} (- \ma{L} - \ma{U})
\end{equation*}
\emph{Matrixdarstellung:}
\begin{equation*}
	\vec{x}^{(k + 1)} = \ma{K}_j \vec{x}^{(k)} + \ma{D}^{-1} \vec{b}
\end{equation*}
\emph{Komponentenweise:}
\begin{equation*}
	x_i^{(k + 1)} = \frac{1}{a_{ii}} \left( b_i - \sum_{j = 1, j \ne i}^n a_{ij} x_{j}^{(k)} \right)
\end{equation*}
\textbf{Vorkonditionierung:}
\begin{equation*}
	\ma P = \ma D^{-1}\quad\Rightarrow\quad\ma P\ma A \vec x = \ma P\vec b
\end{equation*}
\end{sectionbox}

\begin{sectionbox}
\subsection{Gauß-Seidel-Verfahren}
	Unterschied zu Jacobi: Komponentenweise Berechnung von $\vec x$ mit bereits iterierten Werten. (Kürzere Iterationszyklen)
\begin{equation*}
	\ma{K}_{gs} = - (\ma{D} + \ma{L})^{-1} \ma{U}
\end{equation*}
\emph{Matrixdarstellung:}
\begin{equation*}
	\vec{x}^{(k + 1)} = \ma{K}_{gs} \vec{x}^{(k)} + (\ma{D} + \ma{L})^{-1} \vec{b}
\end{equation*}
\emph{Komponentenweise:}
\begin{equation*}
	x_i^{(k + 1)} = \frac{1}{a_{ii}} \left( b_i - \sum_{j = 1}^{i - 1} a_{ij} x_{j}^{(k + 1)} - \sum_{j = i + 1}^{n} a_{ij} x_{j}^{(k)}\right)
\end{equation*}
Konvergiert falls $\rho(\ma{K}_{gs}) < 1$ oder $\ma{A}$ strikt diagonaldominant oder $\ma{A}$ positiv definit \\
Falls $A$ tridiagonal und positiv definit
\begin{equation*}
	\rho(\ma{K}_{gs}) = \rho(\ma{K}_j)^2
\end{equation*}
\textbf{Vorkonditionierung:}
\begin{equation*}
	\ma P = (\ma D + \ma L)^{-1}\quad\Rightarrow\quad\ma P\ma A \vec x = \ma P\vec b
\end{equation*}
\end{sectionbox}

\begin{sectionbox}
\subsection{Successive Over-Relaxation}
\begin{equation*}
	\ma{K}_{SOR} = (\ma{D} + \omega \ma{L})^{-1} (\ma{D}(1 - \omega) - \omega \ma{U})
\end{equation*}
\emph{Matrixdarstellung:}
\begin{equation*}
	\vec{x}^{(k + 1)} = \ma{K}_{SOR} \vec{x}^{(k)} + (\ma{D} + \omega \ma{L})^{-1} \omega \vec{b}
\end{equation*}
	\emph{Komponentenweise:}\\
$x_i^{(k+1)} = \omega a_{ii}^{-1} \left( \vec b_i - \sum\limits_{j=1}^{i-1} a_{ij} x_j^{(k+1)} - \sum\limits_{j = i +1}^{n} a_{ij} x_j^{(k)} \right) + (1 - \omega) x_i^{(k)}$\\
Optimale Konvergenz für
\begin{equation*}
	\omega_\text{opt} = \argmin_{\omega} \rho(\ma{K}_{SOR})
\end{equation*}
Falls $A$ positiv definit und tridiagonal $\Rightarrow \rho(\ma{K}_{gs}) = \rho(\ma{K}_j)^2 < 1$:
\begin{equation*}
	\omega_\text{opt} = \frac{2}{1 + \sqrt{1 - \rho(\ma{K}_j)^2}}
\end{equation*}
\end{sectionbox}

\begin{sectionbox}
\subsection{Gradienten-Verfahren}
\textbf{Voraussetzungen:} $\ma A = \ma A^T$ und $\ma A$ positiv definit
\begin{equation*}
	\Phi(\vec{x}) = \frac{1}{2} \vec{x}^T \ma{A} \vec{x} - \vec{x}^T \vec{b}
\end{equation*}
\begin{equation*}
	\vec{r}^{(k)} = \vec{b} - \ma{A} \vec{x}^{(k)}
\end{equation*}
\begin{equation*}
	\alpha^{(k)} = \frac{\vec{r}^{(k)T} \vec{r}^{(k)}}{\vec{r}^{(k)T} \ma{A} \vec{r}^{(k)}}
\end{equation*}
Optimierung: $\vec{r}^{(k+1)} = \vec{r}^{(k)} - \alpha^{(k)} \ma A \vec{r}^{(k)}$
\emph{Matrixdarstellung:}
\begin{equation*}
	\vec{x}^{(k + 1)} = \vec{x}^{(k)} + \alpha^{(k)} \vec{r}^{(k)}
\end{equation*}
\end{sectionbox}

\section{Matrix Zerlegung}
\begin{sectionbox}
	\subsection{LR-Zerlegung von Matrizen (\textbf{L}ower and \textbf{U}pper)}
Geeignetes Lösungsverfahren für $\ma A \vec x = \vec b$, falls $n < 500$\\
$\ma A = \ma L \cdot \ma R$ \quad mit $\ma R$ obere Dreiecksmatrix ($\rang \ma A = \rang\ma R$)\\

	\subsubsection{Pivotisierung (Spaltenpivotsuche)}
	Permutationsmatrix $\ma P^\top = \ma P^{-1}$ vertauscht Zeilen, damit LR Zerlegung bei 0 Einträgen möglich ist.
	Tausche so, dass man durch die betragsmäßig größte Zahl dividiert (Pivotelement) %Verhindert Auslöschung

	\subsubsection{Rechenaufwand (FLOPS)}
	\begin{tablebox}{ll}
		LU-Zerlegung & $\frac{2}{3}n^3 - \frac{1}{2}n^2 - \frac{1}{6}n$\\
		Vorwärtseinsetzen & $n^2 - n$\\
		Rückwärtseinsetzen & $n^2$\\
	\end{tablebox}
	Bei symmetrischer Matrix für LU Zerlegung halbiert.\\
	Aufwand für Berechnung von $\ma A^{-1}: n^3 + n^2 ∈ \mathcal O(n^3)$
\end{sectionbox}

\begin{sectionbox}
\begin{cookbox}{LR-Zerlegung mit Gaußverfahren $\ma A = \ma L \ma R$; $\ma P^{-1} = \ma P^\top$}
	\item Sortiere Zeilen von $\ma A$ mit $\ma P$ so dass $a_{11} > \ldots > a_{n1}$
	\item Zerlegen von $\ma P \ma A \vec x = \vec b$ zu $\ma L (\ma R \vec x) = \ma L \vec y = \ma P^\top \vec b$ mit\\
		$\underset{\text{(Beispiel für } 2 \times 2)}{\text{Zerlegungsmatrix:}}$ $\ma A = \mat{a & b \\ c & d} \ra \mat{a & b \\ \frac{c}{a} & d - \frac{c}{a} b} = \ma A^*$ \\
		mit den Eliminationsfaktoren $l_{ik} = \frac{a_{ik}}{a_{kk}} \overset{z.B.}{=} \frac{c}{a}$
	\item Für jede Spalte der unteren Dreiecksmatrix wiederholen.\\
		 Für eine $n \times n$ Matrix braucht man $n-1$ Durchläufe
	\item $\ma R = \text{triu}(\ma A^*)$ \quad (obere $\triangle$-Matr. von $\ma A^*$, inkl. Diagonalelem.)
	\item $\ma L = \text{tril}(\ma A^*,-1)+\ma 1$ \quad (untere $\triangle$-Matr. mit $1$en auf Diag.)
	\item \textbf{Vorwärtseinsetzen:} $\ma L \vec y = \vec b$ bzw. $\ma L \vec y = \ma P^\top \vec b$ \quad (Löse nach $\vec y$)
	\item \textbf{Rückwärtseinsetzen:} $\ma R \vec x = \vec y$ \quad (Löse nach $\vec x$)
\end{cookbox}
\end{sectionbox}

\begin{sectionbox}
		\subsection{QR-Zerlegung (existiert immer)}
	$\ma A = \ma Q \ma R$ mit $\ma Q^{-1} = \ma Q^\top$\\
	Berechnung (Verfahren): Housholder (numerisch stabil) , Gram-Schmidt, Givens Rotation.\\
	$\ma A \xrightarrow{EZF} \ma H\ma A \xrightarrow{EZF} \tilde{\ma H} \ma H \ma A = \ma R \Ra \ma A = \ma H^\top \tilde{\ma H}^\top \ma R$\\
	Aufgabe: Finde Vektor $\vec v$ der Senkrecht auf $\ma H$ steht.\\

	\subsubsection*{Lösen von LGSen mit der $Q R$ Zerlegung}
	Bestimme $\vec x$ durch Rückwärtssubsitution aus $\ma R \vec x = \ma Q^\top \vec b$
\end{sectionbox}

\begin{sectionbox}
	\begin{cookbox}{QR-Zerlegung mit Householder-Transformation für $\ma A \in \mathbb R^{m\times n }$}
		\item Setze $\vec a = \vec s_1$ (erste Spalte) und $\vec v = \vec a + \sgn ( a_1) \norm{\vec a} \vec e_1$
		\item Berechne \emph{Householder}-Trafomatrix $\ma H_{\vec v_1} = \ma 1_m - \frac{2}{\vec v^\top \vec v} \vec v \vec v^\top$
		\item Erhalte $\ma A^* = \ma H_{\vec v_1} \ma A$ (ersten Spalte bis auf $a_{11}$ nur Nullen)
		\item Wiederhole für $\ma A^*$ ohne 1. Zeile und Spalte (Untermatr. $\ma A^*_{11}$)\\
			Erweitere $\ma H^*_{\vec v_2}$ oben mit $\ma 1_m$ zu $\ma H_{\vec v_2}$ ($h_{11} = 1$)
		\item Nach $p = \min \eset{m - 1, n}$ Schritten: $\ma H_{\vec v_p} \ma A^* = \ma R$ weil
		\item $\ma Q^\top = \ma H_{\vec v_p} \cdots \ma H_{\vec v_1}$ und  $\ma Q^\top \ma A = \ma R$
	\end{cookbox}
\end{sectionbox}

\begin{sectionbox}
	\subsection{Orthogonalisierungsverfahren nach Gram-Schmidt}
	Berechnet zu $n$ Vektoren $\vec v_i$ ein Orthogonalsystem $\vec b_i$\quad ($i \in [1;n])$
	\begin{equation*}
		\vec b_1 = \vec v_1 \qquad\qquad \vec b_i = \vec v_i - \sum\limits_{k=1}^{i-1} \frac{\vec b_k^\top \bdot \vec v_i}{\vec b_k^\top \bdot \vec b_k} \vec b_k
	\end{equation*}
	Erhalte Ortho\textbf{normal}system durch $\vec b_i' = \vec b_i/\norm{\vec b_i}$\\
	QR-Zerlegung: $\ma A = \ma Q \ma R$ mit $\ma Q = \big[\vec b_1', \ldots , \vec b_n'\big]$ \quad $\ma R = \ma Q^\top \ma A$
\end{sectionbox}

\begin{sectionbox}
	\subsection[Givens Rotation (Jacobi-Rotation)]{Givens Rotation (Jacobi-Rotation) \quad $\ma G^{-1} = \ma G^\top$}
	Die orthogonale Givens-Rotationsmatrix $\ma G$ entspricht der Einheitsmatrix wobei 4 Elemente die Form $\mat{c & s \\ -s & c}$ haben. Die $c$ beliebig auf der Hauptdiagonalen und $s/\!-\!s$ in der gleichen Zeile/Spalte wie die $c$.

	\begin{cookbox}{QR-Zerlegung mit Givens-Rotation für $\ma A \in \mathbb R^{m\times n }$}
		\item Initialisierung: Setze $\ma R = \ma A$ und $\ma G_\text{gesamt} = \ma 1_m$
		\item Wiederhole folgende Schritte für alle Elemente $r_{xy}$ in $\ma R$, welche $0$ werden müssen um obere Dreiecksmatrix zu erhalten. (Reihe $x$, Spalte $y$), verfahre spaltenweise (links nach rechts) und in jeder Spalte von oben nach unten:
		\item Setze $a = r_{yy}$ (Hauptdiagonalelement in dieser Spalte)
		\item Setze $b = r_{xy}$ (Wert, welcher durch $0$ ersetzt werden soll)
		\item Berechne $c := \frac{a}{p}$ und $s := \frac{b}{p}$ mit $p := \sqrt{a^2 + b^2}$
		\item Setze $\ma G = (g_{ij}) =
						\begin{cases}
							c & i = x, j = x \\
							c & i = y, j = y \\
							s & i = y, j = x \\
							-s & i = x, j = y \\
							\text{Einheitsmatrix} & \text{sonst}
						\end{cases}$
		\item Setze $\ma R = \ma G \ma R$ und $\ma G_\text{gesamt} = \ma G \ma G_\text{gesamt}$
		\item Fahre, falls nötig, mit nächstem Element in $\ma R$ fort
		\item Erhalte $\ma Q = \ma G_\text{gesamt}^\top$ \quad Löse
	\end{cookbox}
\end{sectionbox}

\begin{sectionbox}
\subsection{Dünnbesetzte Matrizen}
\parbox{3cm}{\textbf{Ziel:} effizienteres Speichern von Matrizen mit vielen 0 Einträgen.} \qquad
$
\ma A = \begin{bmatrix}
a & 0 & 0 & 0\\
0 & b & c & 0\\
0 & 0 & 0 & d\\
e & 0 & f & 0
\end{bmatrix}
$

\begin{tablebox}{@{}lccc}
				& \textbf{COO} 		& \textbf{CRS} 		& \textbf{CCS}		\\
\cmrule
\texttt{row} 	& $\{1,2,2,3,4,4\}$ & 					& $\{1,4,2,2,4,3\}$	\\
\texttt{rowptr} & 					& $\{1,2,4,5,7\}$ & 					\\
\texttt{col} 	& $\{1,2,3,4,1,3\}$ & $\{1,2,3,4,1,3\}$ & 					\\
\texttt{colptr}	& 					& 					& $\{1,3,5,6,7\}$	\\
\texttt{val} 	& $\{a,b,c,d,e,f\}$ & $\{a,b,c,d,e,f\}$ & $\{a,b,c,d,e,f\}$	\\
\end{tablebox}
\begin{tabularx}{\columnwidth}{@{}lX}
\textbf{COO} & Zeilen und Spaltenindex von \texttt{val}\\
\textbf{CRS} & \texttt{rowptr}(i) zeigt auf j-tes Element von \texttt{col}\\
\textbf{CCS} & \texttt{colptr}(i) zeigt auf j-tes Element von \texttt{row}\\
\end{tabularx}
rowptr(1)=1, rowptr(n)=n+1, gibt an, bei welchem element (in col) die neue Zeile beginnt.
\end{sectionbox}

\section{Numerische Differentiation}
\begin{sectionbox}
\subsection{Vorwärtsdifferenz}
\begin{equation*}
	f'(x_0) \approx \tilde{f}_\text{Vor}'(x_0) = \frac{f(x_0 + h) - f(x_0)}{h}
\end{equation*}
\begin{equation*}
	f'(x_0) - \tilde{f}_\text{Vor}'(x_0) \in \mathcal{O}(h)
\end{equation*}

\subsection{Rückwärtsdifferenz}
\begin{equation*}
	f'(x_0) \approx \tilde{f}_\text{Rück}'(x_0) = \frac{f(x_0) - f(x_0 - h)}{h}
\end{equation*}
\begin{equation*}
	f'(x_0) - \tilde{f}_\text{Rück}'(x_0) \in \mathcal{O}(h)
\end{equation*}

\subsection{Zentrale Differenz}
\begin{equation*}
	f'(x_0) \approx \tilde{f}_\text{Zentral}'(x_0) = \frac{f(x_0 + h) - f(x_0 - h)}{2h}
\end{equation*}
\begin{equation*}
	f'(x_0) - \tilde{f}_\text{Zentral}'(x_0) \in \mathcal{O}(h^2)
\end{equation*}
	$h_{opt} = \sqrt[3]{\frac{3\epsilon}{M}}$ \qquad Max. Rundungsfehler $\epsilon$
\end{sectionbox}

\section{Numerische Integration}
\begin{sectionbox}
\subsection{Polynom-Ansätze}
\begin{equation*}
	\int_a^b f(x) \diff x \approx \int_a^b P(x) \diff x
\end{equation*}

\subsubsection{Lagrange}
\begin{equation*}
	P(x) = \sum_{k = 0}^n L_{n, k}(x)\cdot f(x_k)
\end{equation*}
\begin{equation*}
	L_{n, k}(x) = \prod_{i = 0, i \ne k}^n \frac{x - x_i}{x_k - x_i}
\end{equation*}

\subsubsection{Differenzen}
$f[x_i] = f(x_i)$ \qquad $f[x_i, x_{i+1}] = \frac{f[x_{i + 1}] - f[x_i]}{x_{i+1} - x_i}$
\begin{equation*}
	f[x_i, \dots, x_j] = \frac{f[x_{i + 1}, \dots, x_j] - f[x_i, \dots x_{j - 1}]}{x_j - x_i}
\end{equation*}
\begin{equation*}
	P(x) = f[x_0] + \sum_{k = 1}^n f[x_0, \dots, x_k] (x - x_0) \dots (x - x_{k - 1})
\end{equation*}
\end{sectionbox}

\begin{sectionbox}
\subsection{Newton-Cotes}
\begin{equation*}
	\int_a^b f(x) \diff x \approx \sum_{i = 0}^n g_i f(x_i)
\end{equation*}
\begin{equation*}
	h = \frac{b - a}{n}
\end{equation*}

\subsubsection{Trapez}
falls $n = 1$:
\begin{equation*}
	\int_a^b f(x) \diff x \approx (b - a) \frac{f(a) + f(b)}{2}
\end{equation*}
\textbf{Zusammengesetztes Trapez: } $n = $\#Kanten = \#Stützstellen - 1
\begin{equation*}
	\int_a^b f(x) \diff x \approx  \frac{h}{2} \sum\limits_{k = 0}^{n-1} \left( f(x_k) + f(x_{k+1})  \right)
\end{equation*}

\textbf{Allgemein:}
\begin{equation*}
	\int_a^b f(x) \diff x \approx h \left( \frac{f(a) + f(b)}{2} + \sum_{k = 1}^{n - 1} f(a + k \cdot h) \right)
\end{equation*}
\end{sectionbox}

\begin{sectionbox}
\subsubsection{Simpson $\frac{1}{3}$ (Fassregel)}
falls $n = 2$:
\begin{equation*}
	\int_a^b f(x) \diff x \approx \frac{b - a}{6} (f(a) + 4 f(a + h) + f(b))
\end{equation*}
\textbf{Allgemein} (zusammengesetzte Simpsonregel):
\begin{equation*}
	\int_a^b f(x) \diff x \approx \frac{h}{3} \left( f(a) + f(b) + \sum_{k = 1}^{n - 1} a_k f(a + k \cdot h) \right)
\end{equation*}
\begin{equation*}
	a_k = 3 + (-1)^{k + 1}
\end{equation*}

\subsubsection{Simpson $\frac{3}{8}$}
falls $n = 3$:
\begin{equation*}
	\int_a^b f(x) \diff x \approx \frac{3h}{8} (f(a) + 3 f(a + h) + 3 f(a + 2h) + f(b))
\end{equation*}
\end{sectionbox}

\begin{sectionbox}
\subsection{Kubische Splines $S(x)$}
Stückweise Approximation von $f(x)$ durch $n$ kubische Polynome mit $S(x_i) = f(x_i)$\\
Bestimme Parameter $a,b,c,d$ für jedes Teilstück:
\begin{align*}
		S_j(x) &= a_j + b_j (x-x_j) + c_j (x-x_j)^2 + d_j (x-x_j)^3\\
		S_j'(x) &= b_j+ 2c_j (x-x_j) + 3d_j (x-x_j)^2
\end{align*}
Für $j=0,1,\ldots,n-1$:
\begin{equation*}
	S_j(x_j) = f(x_j) \wedge S_j(x_{j+1}) = f(x_{j+1})
\end{equation*}
Für $j=0,1,\ldots,n-2$:
\begin{align*}
	S_j(x_{j+1}) &\stackrel{!}{=} S_{j+1}(x_{j+1}) ≡ a_{j+1}\\
	S_j'(x_{j+1}) &\stackrel{!}{=} S_{j+1}'(x_{j+1}) ≡ b_{j+1} \\
	S_j''(x_{j+1}) &\stackrel{!}{=} S_{j+1}''(x_{j+1}) ≡ 2c_{j+1}
\end{align*}
Freier bzw. natürlicher Rand:
\begin{equation*}
	S''(x_0) = S''(x_n) = 0
\end{equation*}
Eingespannter Rand:
\begin{equation*}
	S'(x_0) = f'(x_0) \wedge S'(x_n) = f'(x_n)
\end{equation*}
\end{sectionbox}

\begin{sectionbox}
\begin{cookbox}{Parameterbestimmung}
\item $a_j = f(x_j)$ und $h_j = x_{j+1} - x_j$
\item Löse LGS für $\vec c$: $\ma A \vec c = \vec l$\\
\scalebox{.9}{$\ma A = \begin{bmatrix}
1 & 0 & 0 & \dots & 0\\
h_0 & 2(h_0+h_1) & h_1 & \ddots & \vdots\\
\ddots & \ddots & \ddots & \ddots & \vdots\\
\vdots & \ddots & h_{n-2} & 2(h_{n-2}-h_{n-1}) & h_{n-1}\\
0 & \dots & 0 & 0 & 1
\end{bmatrix}$}\\
$\vec l = \begin{bmatrix}
0 \\ \frac{3}{h_1}(a_2 - a_1) - \frac{3}{h_0}(a_1 - a_0) \\ \vdots \\ \frac{3}{h_{n-1}}(a_{n} - a_{n-1}) - \frac{3}{h_{n-2}}(a_{n-1} - a_{n-2}) \\ 0
\end{bmatrix}$
\item $b_j = \frac{1}{h_j}(a_{j+1}- a_j)-\frac{h_j}{3}(2c_j+c_{j+1})$
\item $d_j = \frac{1}{3h_j}(c_{j+1}-c_j)$
\end{cookbox}
\end{sectionbox}

\section{Least Squares}
\begin{sectionbox}
	\subsection{Ausgleichsrechnung}
	Gegeben: $n$ Datenpunkte $(x_i,y_i)$, Gesucht: Eine Polynom-Funktion $f$ welche die Datenpunkte möglichst gut (kleinstes Fehlerquadrat) approximiert.
	Es gilt: $\vec f_{\vec \alpha}(\vec x) = \vec y + \vec r \approx \vec y$ mit Residum $\vec r$\\
	Bestimme $k$ Parameter $\alpha_j$ so, dass Fehlerquadrat $\vec r^\top\vec r$ minimiert wird.\\
	Erstelle $\ma A \in \R^{n \times k}$ mit $\ma A \vec \alpha \approx \vec f(\vec x)$, Zeilen aus $k$ $x$-Termen: $x^2,x,1$\\
	$\min\limits_\alpha \vec r = \min\limits_\alpha \norm{\ma A \vec \alpha - \vec y}_2^2 = \min\limits_\alpha \norm{\vec y - \ma A \vec \alpha}_2^2$\\
	Minimierung durch Ableitung: $\forall j\in[1,k]:\frac{\partial (\vec r)^2}{\partial \alpha_j} \stackrel{!}{=} 0$\\
	Dadurch ergibt sich: $\ma A^\top \ma A \vec \alpha = \ma A^\top \vec y$
	\begin{cookbox}{Lösen der Normalengleichung}
		\item Bestimme eine reduzierte QR-Zerlegung \\ $\ma A = \tilde{\ma Q}  \tilde{\ma R}$ mit $\tilde{\ma Q} \in \mathbb R^{n \times k}, \tilde{\ma R} \in \mathbb R^{k \times k}$
		\item Löse $\tilde{\ma R } \vec x =\tilde{\ma Q}^\top \vec y$
	\end{cookbox}

	\subsubsection{Lineare Ausgleichsrechnung ($k=2$)}
	$f_{\vec \alpha}(x) = \alpha_1 x + \alpha_0$  \qquad $\ma A = [\vec x \quad \vec 1]$ \quad $\vec \alpha = \mat{α_1 \\ α_0}$
	\begin{equation*}
		\argmin_{α_1, α_0} E(α_1, α_0) = \sum_{i = 1}^n \left( y_i - (α_1 x_i + α_0) \right)^2
	\end{equation*}

	\subsubsection{Polynomial Least Squares}
	\begin{equation*}
		f_{\vec \alpha}(x) = P(x, \vec \alpha) = \alpha_k x^k + \ldots + \alpha_1 x + \alpha_0
	\end{equation*}
	\begin{equation*}
		\argmin_{\alpha_0, \ldots, \alpha_{k-1}} E_n(\alpha_0, \ldots, \alpha_{k-1}) = \sum_{i = 1}^{n} \left( y_i - P(\vec \alpha, x_i) \right)^2
	\end{equation*}
	Minimierung durch Ableitung: $\forall i\in[0,k-1]:\frac{\partial E_n}{\partial α_j} \stackrel{!}{=} 0$
\end{sectionbox}

\begin{sectionbox}
\subsection{Anwendung in der linearen Ausgleichsrechnung}
(Minimierung d. Restes)\\
Problem: $\ma A^\top \ma A \vec x = \ma A^\top \vec b$ mit $\ma A \in \mathbb R^{m \times n }$ und $\vec b \in \mathbb R^{m}$ \\
\begin{cookbox}{Lösen der Normalengleichung}
	\item Bestimme eine reduzierte QR-Zerlegung \\ $\ma A = \tilde{\ma Q}  \tilde{\ma R}$ mit $\tilde{\ma Q} \in \mathbb R^{m \times n}, \tilde{\ma R} \in \mathbb R^{n \times n}$
	\item Löse $\tilde{\ma R } \vec x =\tilde{\ma Q}^\top \vec b$
\end{cookbox}
$\norm{\vec b - \ma A \vec x}^2 = \norm{\ma Q^\top (\vec b - \ma A \vec x)}^2 = \norm{\tilde{\vec b} - \tilde{\ma R} \vec x}^2 + \norm{\vec c}^2 \ge \norm{\vec c^2}$
\end{sectionbox}

\section{Numerische Lösung von Differentialgleichungen}
\begin{sectionbox}
\textbf{Ausgangsproblem: DGL}
\begin{equation*}
	\dot{x}(t) = f(x(t))
\end{equation*}
Idee: Anstatt die Funktion $x(t)$ zu bestimmen, wird versucht die Lösung $x(t=t^*)$ für ein bestimmtes $t^*$ zu finden. Man kennt bereits eine Lösung $x(t_0)$ und hangelt sich von dort mit Schritten $x(t_0 + Δtν)$ (Schrittweite $\Delta t$, $ν$-ter Schritt) nach vorne bis man $x(t^*)$ erreicht.
\begin{equation*}
	\hat{x}(\nu) \hat{=} \hat{x}^{(\nu)} = x(t_0 + \Delta t \nu)
\end{equation*}
\begin{equation*}
	\hat{f}(\nu) \stackrel{\wedge}{=} f(t_0 + \Delta t \nu)
\end{equation*}
\end{sectionbox}

\begin{sectionbox}
\subsection{Expliziter Euler}
\begin{equation*}
	\hat{x}^{(\nu + 1)} = \hat{x}^{(\nu)} + \Delta t \cdot \hat{f}\left(\hat{x}^{(\nu)}\right)
\end{equation*}
stabil für $0 < Δt < 2$, instabil für $Δt > 2$
\subsection{Impliziter Euler}
\begin{equation*}
	\hat{x}^{(\nu + 1)} = \hat{x}^{(\nu)} + \Delta t \cdot \hat{f}\left(\hat{x}^{(\nu + 1)}\right)
\end{equation*}
Löse Gleichung nach $\hat{x}^{(\nu + 1)} $

\subsection{Trapez}
\begin{equation*}
	\hat{x}(\nu + 1) = \hat{x}(\nu) + \frac{\Delta t}{2} (\hat{f}(\nu) + \hat{f}(\nu + 1))
\end{equation*}
\end{sectionbox}

\begin{sectionbox}
\subsection[Gear]{Gear $\mathcal O((Δt)^2)$}
\begin{equation*}
	\hat{x}(\nu + 2) = \frac{4}{3} \hat{x}(\nu + 1) - \frac{1}{3} \hat{x}(\nu) + \frac{2}{3} \Delta t \hat{f}(\nu + 2)
\end{equation*}

\subsection{Heun}
\begin{equation*}
	\hat{x}^{[P]}(\nu + 1) = \hat{x}(\nu) + \Delta t \hat{f}(\nu, \hat{x}(\nu))
\end{equation*}
\begin{equation*}
	\hat{x}(\nu + 1) = \hat{x}(\nu) + \frac{\Delta t}{2} \left( \hat{f}(\nu, \hat{x}(\nu)) + \hat{f}(\nu + 1, \hat{x}^{[P]}(\nu + 1)) \right)
\end{equation*}

\subsection{k-Schritt-Adams-Bashforth}
\begin{equation*}
	\hat x(\nu + k) = \hat x(\nu + k - 1) + \Delta t\sum_{i=0}^{k-1}b_{k,i}\hat f(\nu + i)
\end{equation*}
\begin{tabularx}{\columnwidth}{C|CCCCC}
$b_{i,k}$ & $i=0$ & $i=1$ & $i=2$ & $i=3$ \\ \hline
$k=1$ & $1$ & & & \\
$k=2$ & $-\frac{1}{2}$ & $\frac{3}{2}$ & & \\
$k=3$ & $\frac{5}{12}$ & $-\frac{16}{12}$ & $\frac{23}{12}$ & \\
$k=4$ & $-\frac{9}{24}$ & $\frac{37}{24}$ & $-\frac{59}{24}$ & $\frac{55}{24}$
\end{tabularx}
\end{sectionbox}

\begin{sectionbox}
\subsection{Finite Differenzen}
Stützstellen $t_0, \ldots t_n$\\
\textbf{Vorwärtsdifferenz} mit $x(t_n)$ bekannt:\\
$\frac{1}{h} \mat{ -1 & 1 & & \\ & -1 & 1 & \\ & & .. & .. \\ & & & h } \vect{ x(t_0) \\ x(t_1) \\ .. \\ x(t_n)} = \vect{ \dot x(t_0) \\ \dot x(t_1) \\ .. \\ \dot x(t_n)}$\\
\\
\textbf{Rückwärtsdifferenz} mit $x(t_0)$ bekannt:\\
$\frac{1}{h} \mat{ h & & & \\ -1 & 1 & & \\ & .. & .. & \\ & & -1 & 1 } \vect{ x(t_0) \\ x(t_1) \\ .. \\ x(t_n)} = \vect{ \dot x(t_0) \\ \dot x(t_1) \\ .. \\ \dot x(t_n)}$\\
Für zweite Ableitung immer -1, 2, -1 in einer Zeile\\
\end{sectionbox}

\section{Matlab Sample Code}
\begin{lstlisting}
function x = gaussVerfahren(A, b)
    [L, U, P] = LUZerlegung(A);
    [y] = vorwaertsSubstitution(L, P, b);
    [x] = rueckwaertsSubstitution(U, y);
end
\end{lstlisting}

\begin{lstlisting}
function [L, U, P] = LUZerlegung(A)
    n = size(A, 1);
    L = zeros(n, n);
    P = eye(n);

    for i = 1:n-1
        [pivot, pivotIndex] = max(abs(A(i:n, i)));
        pivotIndex = pivotIndex + (i - 1);
        pivot = A(pivotIndex, i);
        Psub = eye(n);
        Psub(:, [i, pivotIndex]) = Psub(:, [pivotIndex, i]);
        A([i, pivotIndex], :) = A([pivotIndex, i], :);
        L([i, pivotIndex], :) = L([pivotIndex, i], :);
        P = Psub*P;
        pivotRow = A(i, i+1:n);
        for j = i+1:n
            factor = A(j, i)/pivot;
            L(j, i) = factor;
            currentRow = A(j, i+1:n);
            A(j, i+1:n) = currentRow - factor*pivotRow;
            A(j, i) = 0;
        end
    end

    U = A;
    L = L + eye(n);
end
\end{lstlisting}

\begin{lstlisting}
function [y] = vorwaertsSubstitution(L, P, b)
    n = size(L, 1);
    y = zeros(n, 1);
    b = P*b;
    y(1) = b(1)/L(1, 1);

    for i = 2:n
        rowSum = L(i, 1:i-1)*y(1:i-1);
        y(i) = (b(i) - rowSum)/L(i, i);
    end
end
\end{lstlisting}

\begin{lstlisting}
function [x] = rueckwaertsSubstitution(U, y)
    n = size(U, 1);
    x = zeros(n, 1);
    x(n) = y(n)/U(n, n);

    for i = n-1:-1:1
        rowSum = U(i, i+1:n)*x(i+1:n);
        x(i) = (y(i) - rowSum)/U(i, i);
    end
end
\end{lstlisting}

\begin{lstlisting}
function [ x_k,r_k,alpha_k ] = conjugateGradientIteration( A,b,x0,N )
    x_k = zeros(length(x0),N+1);
    r_k = zeros(length(x0),N+1);
    p_k = zeros(length(x0),N+1);

    alpha_k = zeros(1,N);
    beta_k = zeros(1,N);

    x_k(:,1) = x0;
    r_k(:,1) = b-A*x0;
    p_k(:,1) = r_k(:,1);

    for i = 1:N
        Ap = A*p_k(:,i);
        alpha_k(i) = (p_k(:,i)'*r_k(:,i))./(p_k(:,i)'*Ap);
        x_k(:,i+1) = x_k(:,i) + alpha_k(i).*p_k(:,i);
        r_k(:,i+1) = r_k(:,i) -alpha_k(i).*Ap;
        beta_k(i) = (Ap'*r_k(:,i+1))./(Ap'*p_k(:,i));
        p_k(:,i+1) = r_k(:,i+1) - beta_k(i).*p_k(:,i);
    end
end
\end{lstlisting}

\begin{lstlisting}
function [x_k,r_k,alpha_k] = gradientIteration(A,b,x0,N)
    x_k = zeros(length(x0),N+1);
    r_k = zeros(length(x0),N);
    alpha_k = zeros(1,N);

    x_k(:,1) = x0;
    for i = 1:N
        r_k(:,i) = b - A*x_k(:,i);
        alpha_k(i) = (r_k(:,i)'*r_k(:,i))./(r_k(:,i)'*A*r_k(:,i));
        x_k(:,i+1) = x_k(:,i) + alpha_k(i).*r_k(:,i);
    end
end
\end{lstlisting}

\begin{lstlisting}
function [Q, R] = householder(A)
    n = size(A, 1);
    identity = eye(n);
    Q = eye(n);

    for i=1:(n-1)
        a = zeros(n, 1);
        a(i:end) = A(i:end, i);
        v = a + sign(a(i))*norm(a)*identity(:, i);
        Qpartial = identity - 2/(v'*v)*(v*v');
        Q = Qpartial*Q;
        A = Qpartial*A;
    end

    R = A;
    Q = Q';
end
\end{lstlisting}

\begin{lstlisting}
function [Q, R] = givensRotation(A)
    n = size(A, 1);
    Q = eye(n);
    R = A;

    for i = 1:(n-1)
        for j = i+1:n;
            G = createGivensRotation(R, j, i);
            Q = G*Q;
            R = G*R;
        end
    end

    Q = Q';
end
\end{lstlisting}

\begin{lstlisting}
function [G] = createGivensRotation(A, row, col)
    a1 = A(col, col);
    a2 = A(row, col);
    p = sqrt(a1*a1 + a2*a2);
    c = a1/p;
    s = a2/p;
    G = eye(size(A, 1));
    G(row, row) = c;
    G(col, col) = c;
    G(row, col) = (-1)*s;
    G(col, row) = s;
end
\end{lstlisting}

\begin{lstlisting}
function [a,b,c,d] = splineParameter(xi,f)
	n = max(size(xi));% Anzahl der Stuetzstellen
	a = f(xi);
    h = zeros(n-1,1);% Schrittweite

    for i=1:n-1
       h(i) = xi(i+1)-xi(i);
    end
    A = sparse(zeros(n,n));% Matrix fuer LGS
    bs = zeros(n,1);% rechte Seite fuer LGS
    for i=2:n-1
       A(i,i) = 2*(h(i)+h(i-1));
       A(i,i-1) = h(i-1);
       A(i,i+1) = h(i);
       bs(i) = (3/h(i))*(a(i+1)-a(i)) - (3/h(i-1))*(a(i)-a(i-1));
    end
    A(1,1) = 1;
    A(n,n) = 1;
    c = A\bs;% Loesung des LGS
    b = zeros(n,1);% Parameter b fuer Splines
    d = zeros(n,1);% Parameter d fuer Splines
    for i=1:n-1
        b(i) = (1/h(i))*(a(i+1)-a(i))-(h(i)/3)*(2*c(i)+c(i+1));
        d(i) = (1/(3*h(i)))*(c(i+1)-c(i));
    end
end
\end{lstlisting}

\section{Blabla Fragen}
\begin{sectionbox}
\begin{enumerate}
	\item \textbf{Nennen Sie einen Vorteil der Dividierten Differenzen gegenüber der Lagrange-Interpolation.}
	\begin{itemize}
		\item geringerer Aufwand
		\item keine komplette Neuberechnung bei neuer Stützstelle
	\end{itemize}
	\item \textbf{Nennen Sie einen Nachteil der Polynominterpolation gegenüber der Spline-Interpolation.}
	\begin{itemize}
		\item Oszillation am Intervallrand $\Rightarrow$ großer Fehler am Rand
	\end{itemize}

	\item \textbf{Nennen Sie zwei Vorteile des Adams-Bashfort-3-Schrittverfahrens gegenüber der Trapez- Methode zum Lösen nichtlinearer Differentialgleichungen.}
	\begin{itemize}
		\item höhere Genauigkeit (lokaler Fehler kleiner bei gleicher Schrittweite)
		\item explizites Verfahren (geringerer Rechenaufwand)
	\end{itemize}

	\item \textbf{Nennen Sie zwei Vorteile des Gauß-Verfahrens gegenüber dem Jacobi-Verfahren.}
	\begin{itemize}
		\item für alle nicht-singulären Matrizen lösbar
		\item geringerer Aufwand, wenn das gleiche Gleichungssystem mit verschiedenen rechten Seiten gelöst werden soll.
	\end{itemize}

	\item \textbf{Nennen Sie drei numerische Integrationsverfahren, die die gleiche (lokale) Fehlerordnung wie das Trapezverfahren besitzen.}
	\begin{itemize}
		\item Gear
		\item Taylor-Verfahren zweiter Ordnung
		\item Zweischritt Adams Bashfort
	\end{itemize}

	\item \textbf{Nennen Sie einen Vorteil des Jacobi-Verfahrens gegenüber dem Gauß-Seidel-Verfahren.}
	\begin{itemize}
		\item leicht parallelisierbar
	\end{itemize}

	\item \textbf{Nennen Sie einen Nachteil des Jacobi-Verfahrens gegenüber dem Gauß-Seidel-Verfahren.}
	\begin{itemize}
		\item langsamere Konvergenz
	\end{itemize}

	\item \textbf{Geben Sie an, welche numerischen Probleme bei Anwendung der Sekantenmethode zur Bestimmung der Nullstelle von $F(x)$ in der Nähe der Nullstelle $x_0$ auftreten können.}
	\begin{itemize}
		\item In der Nähe der Nullstelle ist $F(x^{(k)}) \approx 0$, weshalb in der Iterationsvorschrift näherungsweise der Term $\frac{0}{0}$ auftreten kann. Dementsprechend können Auslöschungsfehler auftreten.
	\end{itemize}
\end{enumerate}
\end{sectionbox}

\section{Sonstiges}
\begin{sectionbox}
	\subsection{Graphen $G = (V,E)$}
	$m$ Knoten (\textbf{v}ertices) $v_i$, $n$ Kanten (\textbf{e}dges) $e_j$\\
	einfach/multi: nur eine/mehrere Kanten zwischen zwei Knoten\\
	gerichtet: Kanten nur in eine Richtung. \quad gewichtet: Kanten haben Werte\\
	Zyklus: Gleicher Start und Endknoten $v_{\ir start} = v_{\ir end}$\\
	Pfad: Alle Knoten verschieden $v_k \ne v_l$\\
	Kreis: Zyklus und Pfad zusammen\\
	\\
	\textbf{Adjazenzmatrix} $\ma A = (a_{ij}) \in \B^{\abs{V} \times \abs{V}}$: \\
	$a_{ij} = \begin{cases} 1 & \text{falls } v_i \text{ mit } v_j \text{ verbunden ist} \\ 0 & \text{sonst} (∄e:(v_i, v_j)∈e) \end{cases}$\\
	$\ma A$ immer symmetrisch, geht nur für ungerichtete Graphen!\\
	\\
	\textbf{Inzidenzmatrix} $\ma B = (b_{ij}) \in \B^{\abs{V} \times \abs{E}}$: \\
	$b_{ij} = \begin{cases} -1 & \text{falls } e_j \text{ bei } v_i \text{ startet } (e_j = (v_i, v_x)) \\ 1 & \text{falls } e_j \text{ bei } v_i \text{ endet } (e_j = (v_x, v_i))\\ 0 & \text{sonst } (v_i ∉ e_j) \end{cases}$\\
	Jede Zeile (für jede Kante) enthält genau einmal $1$ und $-1$\\
	Rang von $\ma B$: $|V| -$\#zusammenhängende Gebiete. Maximal $|V| - 1$!
	Nullraum $\ker \ma B$: Vektoren der Gebiete. Also mindestens $\vec 1 ∈ \ker \ma B$\\
	\\
	\textbf{Laplacematrix} $\ma L = \ma B^\top \ma B = (l_{ij}) \in \B^{\abs{V} \times \abs{V}}$\\
	$l_{ij} = \begin{cases} d_i & \text{falls } i=j \text{ und genau } d_i \text{ Kanten von } v_i \text{weggehen}\\ -1 & \text{falls } i≠j \text{ und die Kante } (v_i, v_j) \text{ existiert }\\ 0 & \text{sonst} \end{cases}$\\
\end{sectionbox}

\begin{sectionbox}
	\subsection{Kirchhoff (Inzidenzmatrix B)}
	KCL: $\ma B^\top \vec i = \vec 0$ \quad KVL: $\vec u - \ma B \vec u_{\ir knoten} = \vec 0$ \quad Ohm: $\vec i = \ma G \cdot \vec u$
\end{sectionbox}

\begin{sectionbox}
	\subsection{Komplexität / Landau-Notation}
	Definiert Zeit und Platzbedarf von Algorithmen ($\exists c \in \R\forall n > n_0$)\\[-1em]
	\begin{tablebox}{l@{\quad $\Ra$ \quad}l}
		$f \in \mathcal O \bigl(g(n)\bigr)$ & $0 \le f(n) \le c \cdot g(n)$\\
		$f \in \Omega \bigl(g(n)\bigr)$ & $f(n) \ge c \cdot g(n) \ge 0$\\
		$f \in \Theta \bigl( g(n) \bigr)$ &  $c_1 \cdot g(n) \le f(n) \le c_2 \cdot g(n)$\\
	\end{tablebox}
	\subsubsection*{Schrankenfunktionen (für große $n \in \N$)}
	$1<\log_{10}(n)<\ln(n)<\log_2(n)<\sqrt{n}<n<n\cdot \ln(n)<(\log n)! <n^2 < e^n < n! < n^n < 2^{2^n}$
\end{sectionbox}

\begin{sectionbox}
	\subsection{Gleitkommadarstellung nach IEEE 754}
	Bitverteilung(single/double):\\
	\begin{tabular}{|c|c|c|} \hline
		$s(1)$ & \quad $e(8/11)$ \qquad & \qquad\qquad\qquad $f(23/52)$ \qquad\qquad\qquad\qquad \\ \hline
	\end{tabular} \\[0.5em]
	$s$: Vorzeichen, $e$: Exponent, $f$: Mantisse\\
	Wert $Z = (-1)^s \cdot  1.f \cdot 2^{e-127}$ \qquad Genauigkeit: $M = 2^{-f}$
\end{sectionbox}

\begin{sectionbox}
	\subsection[Singulärwertszerlegung]{Singulärwertszerlegung $\ma A ∈ \K^{m \times n} = \ma U \ma{Σ} \ma V^\top$}
	Zerlegung in zwei Rotationen und eine Streckung:\\
	$\ma U∈\K^{m\times m}, \ma V∈\K^{n\times n}$: orthonormale Rotationsmatrizen\\
	$\ma{Σ}∈\K^{m\times n}$: $\ma 1\vec{σ}$ ergänzt mit 0en damit $\dim \ma{Σ} = \dim \ma A$

	\begin{cookbox}{Singulärwertszerlegung}
		\item Bestimme $n$ EW $\lambda_i$ von $\ma A^\top \ma A$, sortiere $\lambda_1 \ge \ldots \ge \lambda_n \ge 0$\\ Erhalten $n$ Singulärwerte $σ_i = \sqrt{λ_i}$
		\item Bestimme ONB $\ma V = [\vec v_1,\ldots , \vec v_n]$ aus EV von $\ma A^\top \cdot \ma A$
		\item Bestimme ONB $\ma U = [\vec u_1,\ldots , \vec u_k]$ mit $\vec u_i =\frac{1}{\sigma_i}\ma A \vec v_i$\\ $k= \min(m,n)$, falls $n < m$: Ergänze $\ma U$ zu ONB des $\K^m$
		\item Berechne $\underset{m \times n}{\ma \Sigma}  = \underset{m \times m}{\ma U^\top} \!\cdot\! \underset{m \times n}{\ma A} \!\cdot\! \underset{n \times n}{\ma V}$ \quad ($\ma U, \ma V$ sind orthogonal)
	\end{cookbox}
\end{sectionbox}

Stabilität: Falls Fehler < $κσ\epsilon$ und $σ$ kleiner als ausgeführte Iterationen

\end{document}
